\titlespacing\section{\parindent}{0pt}{12pt}
\section{Анализ предметной области}
\titlespacing\subsection{\parindent}{0pt}{24pt}
\subsection{Виды атак на пароли}

Сейчас существует огромное количество различных атак для того, чтобы получить доступ к информации ограниченного доступа.

Основные виды атак \cite{bib3}:
\begin{itemize}
    \item [---] полный перебор;
    \item [---] подглядывание из-за плеча --- злоумышленник подсматривает набор пароля пользователем и затем использует его сам;
    \item [---] троянский конь --- злоумышленник скрытно устанавливает программу, имитирующую обычный механизм аутентификации, но на самом деле она собирает информацию об именах и паролях пользователей при их попытках входа в систему;
    \item [---] перехват данных в момент их перемещения от пользователя к механизму аутентификации или между устройствами.
 \end{itemize}

\titlespacing\subsection{\parindent}{24pt}{12pt}
\subsection{Базовые понятия}
\titlespacing\subsubsection{\parindent}{0pt}{24pt}
\subsubsection{Хэш-функция}
Хэш-функция \cite{bib4} ---  функция, осуществляющая преобразование массива входных данных произвольной длины в выходную битовую строку установленной длины, выполняемое определенным алгоритмом.

Она является ключевым инструментом защиты пароля при его хранении и передачи на сервер. 
Хэш-функции применяются:
\begin{itemize}
    \item[---] при проверки целостности информации;
    \item[---] при хранении паролей;
    \item[---] в авторском праве (для создания электронно-цифровой подписи);
    \item[---] при решении задачи дедубликации.
\end{itemize}

\clearpage

Криптографическая стойкость \cite{bib5} --- это способность криптографического алгоритма противостоять криптоанализу. Алгоритм считается стойким, когда успешная атака требует от атакующего обладания недостижимым объемом вычислительных ресурсов или значительных затрат времени на раскрытие, что к его моменту информация уже потеряет актуальность.

При рассмотрении хэш-функций под алгоритмом подразумевается процесс вычисления значения хэш-функции, а под атакой на алгоритм --- решение обратной задачи: нахождение для заданного значения хэш-функции \textit{Y1} такого массива входных данных \textit{X1}, что \textit{f(X1) = Y1}. Хэш-функцию, которая является стойкой по определению криптографической стойкости по отношению к такой задаче, называют криптостойкой.

У криптостойких функций имеется следующее свойство: при наличии массива входных данных \textit{X1} и значения хэш-функции для него \textit{f(X1)}, сложной задачей является не только нахождение обратного значения, но и задача нахождения такого отличного от \textit{X1} значения массива входных данных \textit{X2}, для которого верно \textit{f(X1) = f(X2)}. Такие значения \textit{X1} и \textit{X2}, для которых верно равенство \textit{f(X1) = f(X2)}, называются коллизиями.



\titlespacing\subsubsection{\parindent}{24pt}{24pt}
\subsubsection{OTP-токен}
OTP-токен \cite{bib2} --- мобильное персональное устройство, которое принадлежит определенному пользователю и генерирует одноразовые пароли, используемые для аутентификации данного пользователя.

OTP-токены условно можно разбить на две группы \cite{bib10}: либо у пользователя есть физические инструменты для такой генерации, например аппаратные устройства с экраном, мобильные приложения (Google Authenticator, \linebreak Яндекс.Ключ, Aladdin 2FA, Microsoft Authenticator) и т.д., либо пароли приходят по какому-то каналу --- в виде SMS-сообщения, пуш-уведомления и др. 

\subsubsection{Открытый и закрытый ключи}
Открытый и закрытый ключи \cite{bib6} --- парные, зависимые друг от друга ключи. Открытый ключ позволяет кому-либо обмениваться с пользователем шифрованными сообщениями и проверять подлинность его подписи. Закрытый ключ хранится в секрете, позволяет читать зашифрованные сообщения и ставить защищенную от подделки подпись. Ключи в шифровальных системах с открытым ключом --- очень большие числа, иногда из более чем тысячи цифр. Ключи математически связаны так, что, зная открытый ключ, практически невозможно вычислить закрытый ключ.