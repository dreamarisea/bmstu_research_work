\titlespacing\section{\parindent}{0pt}{24pt}
\section*{\hfill{\centering\normalsize ВВЕДЕНИЕ}\hfill}
\addcontentsline{toc}{section}{ВВЕДЕНИЕ}
В современном мире важным вопросом является обеспечение информационной безопасности любых информационных систем, так как в них хранится и обрабатывается большой объем информации ограниченного доступа. Вместе с быстрым развитием информационных технологий активно развиваются и злоумышленные действия над информацией, такие как кража, изменение или удаление данных \cite{bib1}.

Поэтому сейчас защита информации является одной из ключевых задач в информационных системах. В основе защиты информации лежит базовый принцип --- разграничение доступа в систему. Для его соблюдения необходимо проверять подлинность субъекта, получающего доступ к информации. Данный процесс называется аутентификацией \cite{bib2}.

Цель данной работы --- классификация существующих методов аутентификации пользователя.

Для достижения поставленной цели требуется решить следующие задачи:
\begin{itemize}
    \item [---] описать основные виды атак на пароли;
    \item [---] рассмотреть базовые элементы и понятия, используемые при проектировании методов аутентификации пользователя;
    \item [---] провести анализ существующих методов аутентификации пользователя;
    \item [---] провести классификацию методов аутентификации пользователя на основе выделенных критериев.
\end{itemize}